
\begin{parchment}[Jean-Baptiste Joseph Fourier (1768-1830)]{ Fue un matemático y físico francés conocido por sus trabajos sobre la descomposición de funciones periódicas en series trigonométricas convergentes llamadas Series de Fourier, método con el cual consiguió resolver la ecuación del calor. La transformada de Fourier recibe su nombre en su honor. Fue el primero en dar una explicación científica al efecto invernadero en un tratado.
Inició sus estudios en la Escuela Superior Benedictina de Auxerre, orientándose inicialmente a la carrera religiosa, hasta que el monarca Luis XV la convirtió en academia militar. Jean-Baptiste fue seleccionado como estudiante en la institución ya reformada, donde permanecería hasta los 14 años de edad, y empezó a ser instruido en idiomas, música, álgebra y matemáticas, materia en la que destacó, lo que le encaminó a dedicarse al estudio de las ciencias.
Posteriormente, participó en la Revolución francesa y, gracias a la caída del poder de Robespierre, se salvó de ser guillotinado. Se incorporó a la Escuela Normal Superior de París en donde tuvo entre sus profesores a los matemáticos Joseph Louis Lagrange y Pierre Simon Laplace. Posteriormente, ocupó una cátedra como docente en la prestigiosa École polytechnique.
Fourier participó en la expedición de Napoleón Bonaparte a Egipto en 1798. 
%Ya designado secretario perpetuo del Instituto de Egipto el 22 de agosto de 1798, presentó numerosas memorias y dirigió una de las comisiones de %exploración del Alto Egipto. Entre las distintas funciones políticas o administrativas que llevó a cabo, destaca la de comisario francés en el Divan. %A la muerte del General en Jefe del Ejército de Oriente Jean Baptiste Kléber a manos de un fanático sirio en su residencia en El Cairo, Fourier, %quien era amigo y colaborador del General Kléber, es designado para pronunciar el elogio fúnebre, el 17 de junio delante del Instituto de Egipto. A %su regreso a Francia en 1801, Napoleón Bonaparte le nombró prefecto de Isère y entre 1802 y 1815, Fourier presentó a Jean-François Champollion a los %veteranos de la expedición de Egipto.
Entró a la Academia de Ciencias Francesa en 1817 y al cabo de cinco años se convirtió en el secretario perpetuo de las secciones de matemáticas y física.
%Murió en París el 16 de mayo de 1830.
Fue en Grenoble donde condujo sus experimentos sobre la propagación del calor que le permitieron modelar la evolución de la temperatura a través de series trigonométricas. Estos trabajos mejoraron el modelado matemático de fenómenos físicos y contribuyeron a los fundamentos de la termodinámica.
Sin embargo, la simplificación excesiva que proponen estas herramientas fue muy debatida, principalmente por sus maestros Laplace y Lagrange.
Publicó en 1822 su Théorie analytique de la chaleur (Teoría analítica del calor), tratado en el cual estableció la ecuación diferencial parcial que gobierna la difusión del calor solucionándola mediante el uso de series infinitas de funciones trigonométricas, lo que establece la representación de cualquier función como series de senos y cosenos, ahora conocidas como las series de Fourier. El trabajo de Fourier provee el impulso para trabajar más tarde en las series trigonométricas y la teoría de las funciones de variables reales.
%Los dos primeros capítulos de la obra citada tratan problemas sobre difusión de calor entre cuerpos disjuntos en cantidad finita. 
Fourier en esta obra dedujo la ecuación en derivadas parciales que rige tal fenómeno, la cual es conocida como la ecuación del calor.}
%En el capítulo III de la obra, titulado Difusión del calor en un cuerpo rectangular infinito Fourier introduce su método original de trabajo con %series trigonométricas.}
\end{parchment}


\begin{parchment}[
Karl Friedrich Gauss (1777 - 1855)]{ Matemático, físico y astrónomo alemán. Nacido en el seno de una familia humilde, desde muy temprana edad Gauss dio muestras de una prodigiosa capacidad para las matemáticas (según la leyenda, a los tres años interrumpió a su padre cuando estaba ocupado en la contabilidad de su negocio para indicarle un error de cálculo). Durante su vida, se reconoció que era el matemático más grande de los siglos XVIII y XIX. }
\end{parchment}


\begin{parchment}[ Augustin Louis Cauchy ( 1789-1857)]{ Nació en París y murió en una villa cercana a esa misma ciudad. Es autor de trabajos importantes sobre ecuaciones diferenciales, series infinitas, determinantes, probabilidad, grupos de permutación y de física matemática. En 1814 publicó una memoria que se convirtió en el fundamento de la teoría de las funciones complejas. Su trabajo se conoce por su rigor. Publicó 789 artículos y ocupó puestos en a Facultad de Ciencias, en el Colegio de Francia y en la Escuela Politécnica, todos en París. Ha muchos términos y teoremas que llevan su apellido. Fue un realista fiel y vivió en Suiza, Turín y Praga, después de rehusarse a jurar la alianza. Regresó a París en 1838 y recuperó su puesto en la Academia de Ciencias. En 1848 recuperó su lugar en la Sorbona, que mantuvo hasta su muerte.}
\end{parchment}


\begin{parchment}[Viktor Yakovlevich Bunyakovsky (1804-1889)]{ Nació en Bar, Ucrania, y murió en San Petersburgo, Rusia.  Fue profesor en San Petersburgo de 1846 a 1880. Publicó más de 150 trabajos en matemáticas y mecánica. Fue autor de trabajos importantes en teoría de los números, y demostró y publicó la desigualdad de Schwarz en 1859, 25 años antes que Schwarz. También trabajó en geometría y en hidrostática.}
\end{parchment}

\begin{parchment}[Arthur Cayley (1821-1895)] {Fue un matemático británico. Es uno de los fundadores de la escuela británica moderna de matemáticas puras.
Además de su predilección por las matemáticas, también era un ávido lector de novelas, le gustaba pintar, sentía pasión por la botánica y por la naturaleza en general, y era aficionado al alpinismo.
Se educó en el Trinity College de Cambridge. Estudió durante algún tiempo la carrera de leyes con lo que trabajó de abogado durante 14 años, a la vez que publicaba un gran número de artículos. Luego pasó a ser profesor en Cambridge. Fue el primero que introdujo la multiplicación de las matrices. Es el autor del teorema de Cayley-Hamilton que dice que cualquier matriz cuadrada es solución de su polinomio característico.
Dio la primera definición moderna de la noción de grupo.
En combinatoria, su nombre está unido a la fórmula que cuenta los posibles árboles generadores con nodos etiquetados de orden n.
Se llama a veces octavas de Cayley o números de Cayley a los octoniones.
Es el tercer matemático más prolífico de la historia, sobrepasado tan solo por Euler y Cauchy, con aportaciones a amplias áreas de la matemática. En 1889, Cambridge University Press le pidió que preparara sus artículos matemáticos en forma de colección. Siete volúmenes aparecieron con Cayley como editor, pero tras su fallecimiento, el resto de artículos fue editado por Andrew Forsyth, su sucesor en la cátedra de Cambridge. En total los "Collected Mathematical Papers" comprenden trece grandes volúmenes que contienen 967 artículos.}
\end{parchment}



\begin{parchment}[Marie Ennemond Camille Jordan (1838-1922)] {Fue un matemático francés, conocido tanto por su trabajo sobre la teoría de grupos, como por su influyente Curso de análisis (Cours d’analyse). Estudió en la Escuela Politécnica (promoción de 1855). Fue ingeniero de minas y, más tarde, ejerció como examinador en la misma escuela. En 1876 entró como profesor en el Colegio de Francia, sustituyendo a Joseph Liouville.
Su nombre se asocia a un determinado número de resultados fundamentales:
El teorema de la curva de Jordan: un resultado topológico recogido en análisis complejo.
La forma canónica de Jordan en álgebra lineal.
El teorema de Jordan-Holder, que es el resultado básico de unas series de composiciones.
El trabajo de Jordan incidió de manera sustancial en la introducción de la teoría de Galois en la corriente del pensamiento mayoritario. Investigó también los grupos de Mathieu, los primeros ejemplos de grupos esporádicos. Su Tratado de las sustituciones (Traité des substitutions) sobre las permutaciones de grupos fue publicado en 1870.
El 4 de abril de 1881 fue elegido miembro de la Academia de la Ciencia.
De 1885 a 1921 dirige la «Revista de matemáticas puras y aplicadas» (Journal de mathèmatiques pures et apliqués), fundado por Liouville.}
\end{parchment}


\begin{parchment}[Karl Herman Amandus Schwarz (1843-1921)] { Nació en Hermsdorf, Polonia (hoy Alemania), y murió en Berlín. Estudió química en Berlín, pero se cambió a matemáticas, obteniendo el doctorado. Desempeñó puestos académicos en Halle, Zurich y Göttingen. Reemplazó a Weierstrass en Berlín y enseñó alllí hasta 1917. Trabajó en cálculo de variaciones y en superficies mínimas. Su memoria en ocasión del 70 aniversario de Weierstrass contiene, entre otros temas importantes, la desigualdad para integrales que hoy se conoce como desigualdad de Schwarz.}
\end{parchment}

\begin{parchment}[Tullio Levi-Civita (1873-1941)]{ Fue un matemático italiano, famoso por su trabajo sobre cálculo tensorial, pero que también hizo contribuciones significativas en otras áreas de las matemáticas. Era discípulo de Gregorio Ricci-Curbastro, el inventor (algunos dicen co-inventor con Levi-Civita) del cálculo tensorial. Su trabajo incluye artículos fundamentales en matemáticas puras y aplicadas, la mecánica celeste (notable en el problema de los tres cuerpos) e hidrodinámica.
Levi-Civita personalmente ayudó a Albert Einstein a aprender el cálculo tensorial, en el cual Einstein basaría su relatividad general, y que había luchado por dominar. Su libro de texto en cálculo tensorial El Cálculo Diferencial Absoluto (originalmente un conjunto de notas de la conferencia en italiano de coautoría con Ricci-Curbastro) sigue siendo uno de los textos estándares más de un siglo después de su primera publicación, con varias traducciones disponibles}
\end{parchment}





\begin{parchment}[Gabriela González, La cazadora de ondas gravitacionales] {Gabriela González fue portavoz y coordinó durante seis años un equipo de mil especialistas, que trabajó en las detecciones de ondas gravitacionales efectuadas desde el proyecto LIGO (Ondas Gravitacionales con Interferómetro Láser, por sus siglas en inglés). En febrero de 2016 fue una de los cuatro científicos de LIGO que anunciaron la primera observación ondulatoria gravitacional, detectada en septiembre de 2015. Egresada de la Universidad Nacional de Córdoba y actual profesora en el departamento de física y astronomía de la Universidad de Louisiana, fue reconocida en 2016 como una de los diez científicos más destacados del mundo por la revista académica Nature. Además, a partir de 2018 forma parte de la Academia de Ciencias de Estados Unidos, institución de máximo prestigio internacional. }
\end{parchment}

\begin{parchment}[Luis Ángel Caffarelli, y sus ecuaciones diferenciales] {Nacido el 8 de diciembre de 1948 en Buenos Aires. En 2023 la Academia Noruega de Ciencias le concedió el Premio Abel, el cuál es semejante al nobel en matemáticas, puesto que éste último no cuenta con distinciones para esta rama del conocimiento. Es el principal experto mundial en problemas de frontera libre para ecuaciones diferenciales en derivadas parciales no lineales. También es famoso por sus contribuciones a la ecuación Monge-Ampere y más en general ecuaciones completamente no lineales. Recientemente se ha interesado por los problemas de homogeneización.}
\end{parchment}

\begin{parchment}[Ariel Goobar, El universo se esta acelerando]{Profesor de astrofísica en la Universidad de Estocolmo e integrante del equipo que ganó el Nobel de Físca en 2011 por descubrir mediante las mediciones con supernovas, la expansión acelerada del universo., actualmente es Director del Oskar Klein Centre for Cosmoparticle Physics. Emigrado de Argentina a los 13 años, nació en Córdoba. Se dedica al contenido de lo que se llama materia oscura.
La gran esperanza –aunque a lo mejor me equivoco dijo– desde el punto de vista experimental es que si entendemos qué es la energía oscura, muy probablemente eso nos vaya a dar un indicio importante de cómo se puede entender la fuerza de gravedad, cómo casar la teoría de la relatividad con la física cuántica. Muy probablemente estén relacionadas, pero hoy no tenemos idea de cómo. Hoy, si yo pudiera soñar algo, creo que es entender eso. Sería alucinante. }
\end{parchment}

\begin{parchment}[Juan Martín Maldacena, y sus supercuerdas.]
{Nacido en Buenos Aires, 10 de septiembre de 1968. Entre sus muchos aportes al campo de la teoría de supercuerdas —o Teoría M—, se encuentra la denominada «conjetura de Maldacena», «dualidad de Maldacena» o correspondencia AdS/CFT, que propone la equivalencia entre ciertas teorías de gravedad cuántica y cualquier teoría conforme de campos bajo determinadas condiciones que satisfacen el principio holográfico. En 1997 se unió a la Universidad de Harvard como profesor asociado —entonces el profesor asociado vitalicio más joven de la historia de Harvard—. Ahí en 1999 ascendió a profesor titular. En 2012 fue honrado con el nuevo Premio Yuri Milner a la física fundamental. La distinción le dotó con tres millones de dólares. En ese momento sus investigaciones estaban orientadas a la relación entre espacio y tiempo cuánticos y a las teorías de partículas. En 2018 recibió la Medalla Lorentz siendo así el único científico de habla hispana y de Iberoamérica en haberla recibido.}
\end{parchment}


\begin{parchment}[Alicia Dickestein, (1955)]{En 1982, obtuvo el título de Doctora en Ciencias Matemáticas en la Facultad de Ciencias Exactas y Naturales de la Universidad de Buenos Aires misma institución, con una tesis sobre geometría analítica compleja.
Fue la primera directora del Departamento de Matemática de la Facultad de Ciencias Exactas y Naturales de la UBA (en el período 1996-1998), donde desde 2009 pasó a ser profesora regular titular plenaria. Allí se desempeña como investigadora superior del CONICET, organismo en el que ingresó en 1985.
En la actualidad Dickenstein es Investigadora Superior del Consejo Nacional de Investigaciones Científicas y Técnicas de Argentina (CONICET). En los últimos años concentró su trabajo en las aplicaciones de la geometría algebraica en el ámbito de la biología molecular.
En 2019 se incorporó como Académica Titular de la Academia Nacional de Ciencias Exactas, Físicas y Naturales (ANCEFN). 
A lo largo de su vida participó de una gran cantidad de congresos y reuniones científicas. Dirigió numerosas tesis de licenciatura, maestría y doctorado. Se desempeñó también como jurado de diversas tesis doctorales y de maestría tanto en su país como en el extranjero.
Publicó varios libros, entre los cuales se encuentra Mate max: la matemática en todas partes, que presenta problemas matemáticos destinados a niñas y niños de los últimos años de educación básica.
Fue merecedora del Premio TWAS (de la Academia Mundial de Ciencias) en el área de Matemática, en el año 2015, por su "destacada contribución a la comprensión de discriminantes”.
En 2017 le fue entregado el Premio Consagración en Matemática de la Academia Nacional de Ciencias Exactas, Físicas y Naturales.
En 2021 recibió el premio L'Oréal-UNESCO a Mujeres en Ciencia por su trayectoria en geometría algebraica.
También en 2021 le fue otorgado el reconocimiento como "Personalidad destacada de la Universidad de Buenos Aires", durante los festejos por el Bicentenario de dicha universidad, recibiendo también una medalla personalizada, una moneda acuñada por la Casa de la Moneda y un sello postal del Correo Argentino.}
\end{parchment}

\begin{parchment}[William Gilbert Strang (1934)] {Es un matemático estadounidense, actualmente Professor Mathworks de Matemáticas del Department of Mathematics del Massachusetts Institute of Technology (MIT). Ha contribuido a la teoría de elementos finitos, el calculo de variaciones, análisis wavelets y álgebra lineal. El profesor Strang ha contribuido enormemente a la educación en matemáticas, en forma de libros técnicos y cursos online. En MIT enseña Álgebra lineal, Ciencia Computacional e Ingeniería, Aprendiendo de los Datos. Sus clases están disponibles en la plataforma MIT OpenCourseWare (en inglés).
Gilbert Strang nació en Chicago, Illinois. Cursó estudios en el propio MIT y en el Balliol College, en la Universidad de Oxford. Se doctoró en la Universidad de California, Los Ángeles (UCLA) y desde ese momento ha llevado a cabo su actividad docente en el MIT. También ha sido un Sloan Fellow. Entre las publicaciones más notables del Professor Strang se destaca "An Analysis of the Finite Element Method", conjuntamente con George Fix, así como seis manuales:
Introduction to Linear Algebra (1993, 1998, 2003),
Linear Algebra and Its Applications (1976, 1980, 1988, 2005),
Introduction to Applied Mathematics (1986),
Calculus (1991),
Wavelets and Filter Banks, con Truong Nguyen (1996),
Linear Algebra, Geodesy, and GPS, con Kai Borre (1997).
Gilbert Strang fue Presidente de SIAM (Society for Industrial and Applied Mathematics) durante los años 1999-2000. También ha sido Chairman of the US National Committee on Mathematics durante los años 2003-2004. Es Honorary Fellow, en el Balliol College de Oxford. También es Chairman, en la National Science Foundation (NSF) del Advisory Panel del área de Matemáticas.
Fue pionero al abrir sus clases y permitir que fueran grabadas en vídeo mientras explicaba matemáticas a sus alumnos del MIT para su difusión abierta y gratuita en Internet.}
\end{parchment}
%En el caso particular que el  cambio de las coordenadas $x_i$ a las $x^\prime_i$ sea entre sistemas  
%cartesianos y ortogonales, se tendrá además que 
	
%\bigskip
	
%\begin{equation}
%\label{xxp}	
% x_i= a^h_{i}x^\prime_h
%\end{equation}

%\bigskip

%\noindent
%ya que, en ese caso $A.A^{T}=A^{T}.A=I$ (la inversa de la matriz $A$ es su traspuesta, recordar la matriz de %rotación). 


%Con la nueva notación, en ese caso,  se tiene que

%$$ a^h_{i}(a^j_{h})^t= a^h_{i}a^h_{j}=\delta_{ij}$$ 
%\noindent
%o, equivalentemente,

%$$(a^h_{i})^ta^j_{h}=a^i_{h}a^j_{h}= \delta_{ij}. $$

%\bigskip


\begin{parchment}[Ingrid Daubechies (1954)] {Es una matemática y física belga. Ha realizado importantes aportaciones en el campo de las ondículas en imágenes. En 2020 recibió el Premio Princesa de Asturias de Investigación. En 2023 le conceden el premio Wolf en matemáticas, y es la primera mujer que lo ha recibido.
Ingrid Daubechies nació en Houthalen (Bélgica). Estudió física en la Vrije Universiteit Brussel (la universidad de Bruselas en lengua flamenca), en la que también se doctoró en física teórica en 1980 y estuvo investigando hasta 1987. Ese año se trasladó a Estados Unidos con su marido, el también matemático Robert Calderbank, recién casados. Daubechies trabajó en los Laboratorios Bell de Nueva Jersey y en varias universidades estadounidenses. En 1993 se convirtió en profesora de matemática computacional en la Universidad de Princeton hasta 2011, cuando trasladó a la Universidad Duke como catedrática de matemáticas.
En 2012, el rey Alberto II de Bélgica la concedió el título de Baronesa en reconocimiento de su trayectoria profesional.
Es miembro de numerosas instituciones. Fue la primera mujer matemática en presidir la Unión Matemática Internacional (desde 2011). En 1993 fue admitida en la Academia Estadounidense de las Artes y las Ciencias, en 1998 en la Academia Nacional de Ciencias de Estados Unidos y en 2012 en la Sociedad Estadounidense de Matemática. Además, ha sido invitada a participar en numerosas ocasiones en el Congreso Internacional de Matemáticas.
Daubechies ha recibido numerosos premios, entre ellos destacan el Premio Nemmers en Matemáticas de 2012 y el Premio Fundación BBVA Fronteras del Conocimiento en Ciencias Básicas 2012 junto a David Mumford.
En 2020 fue reconocida, junto a Emmanuel Candès, Yves Meyer y Terence Tao, con el Premio Princesa de Asturias de Investigación Científica y Técnica por «haber realizado contribuciones pioneras y trascendentales a las teorías y técnicas modernas del procesamiento matemático de datos y señales».
En 2023 recibió el Premio Wolf en Matemáticas, por sus investigaciones sobre ondículas y análisis armónico aplicado. Daubechies es la primera mujer que ha recibido este reconocimiento.
Ingrid Daubechies ha trabajado en el campo de las ondículas, herramientas que permiten el análisis de señales para entregar información temporal y frecuencial de manera casi simultánea. En 1988, Daubechies propuso la ondícula ortogonal con soporte compacto (conocida como ondícula Daubechies), y en 1992 la ondícula biortogonal, también conocida como ondícula CDF (Cohen-Daubechies-Feauveau), empleada para el formato de compresión de imágenes JPEG 2000.
Estas herramientas matemáticas permiten el avance e investigación tanto en matemática teórica como aplicada, pues sirve en la demostración tanto de teoremas como en el desarrollo de las telecomunicaciones, tanto en audio como vídeo, y hasta el ámbito biosanitario, con transmisión de datos de imágenes sanitarias.}
\end{parchment}


\begin{parchment}[William Rowan Hamilton (1805-1865)]{ Fue un matemático británico. Es uno de los fundadores de la escuela británica moderna de matemáticas puras e hizo importantes contribuciones al desarrollo de la óptica, la dinámica, y el álgebra. Su descubrimiento del cuaternión, junto con su sistematización de la dinámica, son sus trabajos más conocidos. Este último trabajo sería decisivo en el desarrollo de la mecánica cuántica, donde un concepto fundamental llamado hamiltoniano lleva su nombre.
Hamilton fue el cuarto de los nueve hijos. Vivían en Dublín. Se dice que Hamilton demostró un inmenso talento a una edad muy temprana. 
Su tío observó que Hamilton, desde una edad temprana, había mostrado una asombrosa habilidad para aprender idiomas. A la edad de siete años, ya había hecho un progreso considerable con el hebreo, y antes de los trece años, bajo la supervisión de su tío (un lingüista), había adquirido conocimientos casi en tantos idiomas como años de edad tenía (idiomas europeos clásicos y modernos, y persa, árabe, hindustaní, sánscrito e incluso maratí y malayo). Conservó gran parte de su conocimiento de idiomas hasta el final de su vida, a menudo leyendo persa y árabe en su tiempo libre, aunque hacía tiempo que había dejado de estudiar idiomas y los usaba solo para relajarse.
Hamilton es reconocido como uno de los científicos más destacados de Irlanda, y a medida que la nación se vuelve más consciente de su herencia científica, cada vez se lo celebra más. Se dice que se le permitía pisar el césped de la Universidad, algo totalmente prohibido. Este hecho camina entre la realidad y la ficción. Posiblemente ocurriera que, absorto en sus meditaciones, descuidara esta prohibición y accidentalmente caminase por los jardines, aunque absolutamente nadie en toda Irlanda se hubiera atrevido a interrumpirle o a amonestarle. Esta anécdota seguramente sirve para dar idea de la categoría de Hamilton como uno de los grandes matemáticos de su tiempo y de la historia.
El Instituto Hamilton está dedicado a la investigación sobre matemáticas aplicadas en la Universidad Maynooth. 
Irlanda emitió dos sellos conmemorativos en 1943 para celebrar el centenario del anuncio de los cuaterniones.El Banco Central de Irlanda acuñó en 2005 una moneda de plata conmemorativa de 10 euros para conmemorar los 200 años desde su nacimiento.
Los talleres de mantenimiento más nuevos del sistema de tranvías de Dublín (LUAS), llevan su nombre.
En su juventud, Hamilton tuvo un telescopio y se convirtió en un experto en el cálculo de fenómenos celestes, como por ejemplo, la determinación de la visibilidad de los eclipses de luna. Como había recibido calificaciones extremadamente altas tanto en Clásicos como en Ciencias, no era demasiado inusual que, el 16 de junio de 1827, con solo 21 años y todavía estudiante, fuera elegido Astrónomo Real de Irlanda y se instalara en el Observatorio de Dunsink, donde permaneció hasta su muerte en 1865.
En sus primeros años en Dunsink, Hamilton observó los cielos con bastante regularidad. La astronomía observacional en esos días consistía principalmente en medir las posiciones de las estrellas, lo que no era demasiado interesante para una mente matemática. Pero la razón principal por la que finalmente cedió la observación regular por completo a su asistente de astronomía, Charles Thompson, fue que Hamilton sufría con frecuencia enfermedades después de dedicarse a la observación.
Hoy en día, Hamilton no es reconocido como un gran astrónomo, aunque durante su vida si gozó de esta consideración.24​ Sus conferencias de introducción a la astronomía fueron famosas; además de sus alumnos, atrajeron a muchos eruditos y poetas, e incluso a damas; en aquellos días una hazaña notable. La poetisa Felicia Hemans escribió su poema "La oración del estudiante solitario" después de escuchar una de sus conferencias.}
\end{parchment}



\begin{parchment}[Arthur Cayley (1821-1895)]{ Fue un matemático británico. Es uno de los fundadores de la escuela británica moderna de matemáticas puras.

Además de su predilección por las matemáticas, también era un ávido lector de novelas, le gustaba pintar, sentía pasión por la botánica y por la naturaleza en general, y era aficionado al alpinismo.

Se educó en el Trinity College de Cambridge. Estudió durante algún tiempo la carrera de leyes con lo que trabajó de abogado durante 14 años, a la vez que publicaba un gran número de artículos. Luego pasó a ser profesor en Cambridge. Fue el primero que introdujo la multiplicación de las matrices. Es el autor del teorema de Cayley-Hamilton que dice que cualquier matriz cuadrada es solución de su polinomio característico.

Dio la primera definición moderna de la noción de grupo.

En combinatoria, su nombre está unido a la fórmula  que cuenta los posibles árboles generadores con nodos etiquetados de orden n.
Es el tercer matemático más prolífico de la historia, sobrepasado tan solo por Euler y Cauchy, con aportaciones a amplias áreas de la matemática. En 1889, Cambridge University Press le pidió que preparara sus artículos matemáticos en forma de colección. Siete volúmenes aparecieron con Cayley como editor, pero tras su fallecimiento, el resto de artículos fue editado por Andrew Forsyth, su sucesor en la cátedra de Cambridge. En total los "Collected Mathematical Papers" comprenden trece grandes volúmenes que contienen 967 artículos.}
\end{parchment}














\begin{exercise}
\item

Determine si las siguientes funciones son o no productos internos. En caso afirmativo encuentre su matriz en la base canónica del espacio correspondiente.

a) $\Phi:\mathbb{R}^2 \times \mathbb{R}^2  \rightarrow \mathbb{R}, \Phi ((x_1,x_2),(y_1,y_2))= x_1y_1-x_1y_2-x_2y_1+3x_2y_2$

b) $\Phi:\mathbb{C}^2 \times \mathbb{C}^2  \rightarrow \mathbb{C}, \Phi ((x_1,x_2),(y_1,y_2))= 2x_1 \overline y_1+x_2 \overline y_2-x_1 \overline y_2+x_2 \overline y_1$
\end{exercise}







2 








\begin{exercise} 
\item
Sea $T\in L(V)$, $V$ de $dim<\infty$. Demuestre que si $T$ es diagonalizable y  $\lambda_1,\lambda_2, \cdots,\lambda_k$ son  los autovalores distintos de $T$,  entonces existen  $E_1,E_2, \cdots,E_k$ aplicaciones lineales tales que 

\begin{enumerate}

\bigskip

\item   $E_1+E_2+ \cdots+E_k=I$ 

\bigskip

\item $T=\lambda_1E_1+\lambda_2E_2, \cdots\lambda_kE_k$ 

\bigskip

\item $E_i oE_j=\vec{0}$ $i\neq j$

\bigskip

\item $E_i ^2=E_i$, $1=1,2,\cdots,k$

\bigskip

\item $Im(E_i) =\bold E_{\lambda_i}$

\end{enumerate}

\end{exercise} 